%
% header.tex
%
\documentclass[%
	pdftex,%			PDFTex verwenden
	a4paper,%			A4 Papier
	oneside,%			Einseitig
	idxtotoc,%			Index ins Verzeichnis einfuegen
	parskip,%			Europaeischer Satz mit abstand zwischen Absaetzen oder halfparskip
	headsepline,%      	Linie nach Kopfzeile
	footsepline,%       Linie vor Fusszeile
	11pt%               Groessere Schrift, besser lesbar am Bildschrim
]{scrartcl}

\usepackage{a4wide}
\usepackage[left=3.0cm, right=2.5cm, top=3.0cm, bottom=3.5cm]{geometry}

\usepackage{nofloat}
%
% Paket fuer die Indexerstellung.
%
%\usepackage{makeidx}

%\usepackage{setspace}


%
% Paket fuer Deutsche Trennungen, Anfuehrungsstriche und mehr...
%
\usepackage{german, ngerman}
\usepackage[german]{babel}
%
%Eingabe von ‰,ˆ,¸ erlauben
%
\usepackage[utf8]{inputenc}
\usepackage[T1]{fontenc}

\usepackage[babel,german=quotes]{csquotes}

    \usepackage{color}                                            %%
    \usepackage{longtable}                                        %%
    \usepackage{calc}                                             %%
    \usepackage{multirow}                                         %%
    \usepackage{hhline}                                           %%
    \usepackage{ifthen}  

%
% Paket zum Erweitern der Tabelleneigenschaften
%
\usepackage{array}

% 
% Optisch schoenere Tabellen
%
\usepackage{booktabs}

%
% Farbige Tabellen und longtable + supertabular
%
\usepackage{colortbl}
\usepackage{xcolor}

%
% Ermoeglicht 90° gedrehte Spaltenueberschriften
%
\usepackage{tabularx}
%\usepackage{rotating}
%\newcolumntype{R}[1]{%
%  >{\begin{turn}{90}\begin{minipage}{#1}\scriptsize\raggedright\hspace{0pt}}l%
%  <{\end{minipage}\end{turn}}%
%}

%
% Schriftgroessen manipulieren
%
\usepackage{relsize}

%
% Strichvariationen bei Tabellen
%
%\usepackage{hhline}


%
% Paket um Grafiken einbetten zu koennen
%
\usepackage{graphicx}

%
% Zeilenumbruch bei Bildbeschreibungen.
%
\setcapindent{1em}

%
% Paket um Listings sauber zu formatieren.
%
\usepackage[savemem]{listings}
\lstloadlanguages{TeX}

\usepackage{color}
\definecolor{nb}{gray}{.85}


%
% Fuer den Quellcode
%
\usepackage{listings}
\lstset{numbers=left, numberstyle=\tiny, stepnumber=1, showstringspaces=false}
\lstset{backgroundcolor=\color{nb}}
\lstset{language=C++}


%
% mathematische symbole aus dem AMS Paket.
%
\usepackage{amsmath}
\usepackage{amssymb}

\setcounter{secnumdepth}{4}
\setcounter{tocdepth}{4}
%
% Type 1 Fonts fuer bessere darstellung in PDF verwenden.
%
%\usepackage{mathptmx}           % Times + passende Mathefonts
%\usepackage[scaled=.92]{helvet} % skalierte Helvetica als \sfdefault
\usepackage{courier}            % Courier als \ttdefault

%
% Paket um Textteile drehen zu koennen
%
\usepackage{rotating}

%
% Package fuer Farben im PDF
%
\usepackage{color}

%
% Paket fuer Links innerhalb des PDF Dokuments
%
\definecolor{LinkColor}{rgb}{0,0,0}
\usepackage[plainpages=false,hyperfootnotes=false]{hyperref}
\hypersetup{colorlinks=true,%
	linkcolor=LinkColor,%
	citecolor=LinkColor,%
	filecolor=LinkColor,%
	menucolor=LinkColor,%
	pagecolor=LinkColor,%
	urlcolor=LinkColor}

%
% Index erzeugen
%
\makeindex

%
% EOF
%