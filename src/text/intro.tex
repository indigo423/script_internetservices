\section{Einleitung und Lernziele}
Ziel des Labs ist es das Zusammenspiel grundlegender Internetdienste zu
veranschaulichen. Für die erfolgreiche Durchführung der Übung sind folgende
Voraussetzungen von Vorteil:
\begin{itemize}
  \item Grundkenntnisse mit Linux (Debian/Ubuntu)
  \item Grundkenntnisse im Bereich TCP/IP, Routing und Netztopologien
  \item Grundkenntnisse von verteilten Systemen, Client-/Server Prinzipien
\end{itemize}

In der Übung soll die Einführung und Inbetriebnahme von einfachen
Internetdiensten praktisch vermittelt werden. Aufbauend auf einem Linux-Server
Netzwerk werden DNS, Mail und Webdienste eingerichtet und deren Konzepte
vertieft. Die Übung wird in den folgenden Schritten durchgeführt:
\begin{itemize}
  \item Installation und Einrichtung von zwei virtuellen Linux Servers mit
  Ubuntu unter VMware.
  \item Einrichtung der Netzwerkparameter
  \item Konzeption und Einrichtung einer lokalen DNS-Subdomain mit BIND
  \item Konzeption und Einrichtung einer übergeordneten Domain und das Prinzip
  der "`Domain Delegation"'
  \item Konzeption und Einrichtung einer lokalen Mail-Domain mit Postfix
  \item Vertiefung der Konzepte von DNS im Zusammenspiel mit Mail-Diensten
  \item Konzeption und Aufbau einer Intranet Website auf Basis von Apache2
  \item Fehleranalyse im Umgang mit Linux-Systemen in TCP/IP-Netzwerken
  \item {\bf Optional:} Einrichtung des Labs mit IPv6
\end{itemize}

Diese Laborumgebung wird mit den Studierenden gemeinsam aufgebaut und
eingerichtet. Für jedes IP-Subnetz konfiguriert jede Lerngruppe eine eigene
DNS-Zone mit einem Intranet Web- und Mailserver. Der Internetzugang soll durch
einen zentralen Router mittels NAT realisiert werden. Zusätzlich sollen
DNS-Anfragen ins Internet über einen zentralen Router durchgeführt werden. Im
Abschluss werden erweiterte Konfigurationen für Mail-, Web- und DNS-Dienste
konfiguriert und eingerichtet.

