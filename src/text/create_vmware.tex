
\section{Vorbereitung}
In dieser Übung soll das Zusammenspiel grundlegender Internetdienste veranschaulicht werden.
Hierfür werden zwei virtuelle Maschinen benötigt, die im Folgenden erstellt
werden. Die Speicherorte und Bezeichnung können Sie aus Tabelle
\ref{tab:install-location} entnehmen.

\scriptsize
\begin{table}[!h]
  \centering
	\begin{tabular}{l l l l l}
		\hline
		Name & Pfad & Hostname & Benutzername & Passwort \\
		\hline
		srv1-0 & D:$\backslash$Austausch$\backslash$Internetdienste$\backslash$srv1-0 &
		srv1-0 & admin1 & password1 \\
		srv2-0 & D:$\backslash$Austausch$\backslash$Internetdienste$\backslash$srv2-0 &
		srv2-0 & admin2 & password2 \\
		\hline
	\end{tabular}
	\caption{Pfade für die Installation und Einrichtung der virtuellen Maschinen}
	\label{tab:install-location}
\end{table}
\normalsize 

\subsection{Erzeugung der virtuellen Maschinen}
Im Folgenden wird die Erzeugung einer virtuellen Maschine mit VMware-Workstation exemplarisch für \textbf{srv1} gezeigt. Verfahren Sie für die Maschine \textbf{srv2} analog.
Öffnen Sie das Programm  \textbf{VMware Workstation} und wählen Sie \textbf{File$\rightarrow$New$\rightarrow$Virtual Machine} im Menü.

Verfahren Sie analog für den Server \textbf{srv2}.