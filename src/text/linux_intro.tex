%------------------------------------------------------------------------------
\section{Einführung Linux}
%------------------------------------------------------------------------------
Der Laboraufbau erfordert Kenntnisse im Umgang mit dem Betriebssystem Linux. Im
Folgenden Abschnitt werden einige grundlegende Kommandos beschrieben die für die
erfolgreiche Durchführung des Laborversuchs hilfreich sind. Im Labor wird die
Distribution "`Ubuntu Server"' verwendet. Die hier beschriebene Einführung hat
keinen Anspruch auf Vollständigkeit.

%------------------------------------------------------------------------------
\subsection{System Kommandos}
%------------------------------------------------------------------------------
Für den Laboraufbau sind die folgenden Kommandos zur Fehlerdiagnose und
Überprüfung hilfreich.

Anzeigen der ausführlichen Hilfe eines Programms
\begin{lstlisting}
man <programm> 
\end{lstlisting}

Anzeigen der kurzen Hilfe eines Programms
\begin{lstlisting}
<programm> -h | --help
\end{lstlisting}

Anzeige aller laufenden Systemprozesse inklusive Benutzer, CPU, Memory und
Kommando
\begin{lstlisting}
ps aux 
\end{lstlisting}

Auslastung des Systems anzeigen
\begin{lstlisting}
top
\end{lstlisting}

Anzeige der aktuellen Netzwerkkonfiguration optional mit interface name
$[$ifname$]$ (e.g. eth0)
\begin{lstlisting}
ifconfig [ifname]
\end{lstlisting}

Anzeige der lokalen Routing einträge
\begin{lstlisting}
route -n
\end{lstlisting}

Testen der Layer-3 Verbindung
\begin{lstlisting}
ping <ip-adresse>
\end{lstlisting}

Testen der Namensauflösung
\begin{lstlisting}
nslookup <hostname>
\end{lstlisting}

Nachverfolgen wie die Namensauflösung auf dem System ausgeführt wird
\begin{lstlisting}
dig +trace <hostname>
\end{lstlisting}

Anfrage zur Namensauflösung an einen speziellen Nameserver
\begin{lstlisting}
dig @<namserver> <hostname> 
\end{lstlisting}

Anzeige welche Serverdienste aktuell auf TCP Ports lauschen
\begin{lstlisting}
netstat -lnpt 
\end{lstlisting}

Anzeige welche Serverdienste aktuell auf UDP Ports lauschen
\begin{lstlisting}
netstat -lnpu 
\end{lstlisting}

Anzeige der bestehenen Netzwerkverbindungen auf einem Server
\begin{lstlisting}
netstat -an | grep EST 
\end{lstlisting}

Aufzeichnen des Netzwerkverkehrs auf einem bestimmten Netzwerkinterface
\begin{lstlisting}
tcpdump -i [interface] -w [file]
\end{lstlisting}

Verfolgen einer Router im Netzwerk
\begin{lstlisting}
traceroute [ip-adresse | hostname] 
\end{lstlisting}

Prüfen welche Ports auf einem Netzwerkserver erreichbar sind
\begin{lstlisting}
nmap 
\end{lstlisting}

Serverdienste für den Systemstart in Standardrunlevel hinzufügen
\begin{lstlisting}
update-rc.d <service> defaults 
\end{lstlisting}

Serverdienste für den Systemstart entfernen
\begin{lstlisting}
update-rc.d [-f] <service> remove
\end{lstlisting}

Diverse Editoren
\begin{lstlisting}
vi
vim
nano
pico
\end{lstlisting}

Kennwort des aktuell angemeldeten Benutzers ändern
\begin{lstlisting}
passwd
\end{lstlisting}

%------------------------------------------------------------------------------
\subsection{Verzeichnisse und Konfigurationsdateien}
%------------------------------------------------------------------------------
Für den Laborversuch sind die folgenden Verzeichnisse relevant. Die
Verzeichnisse können bei unterschiedlichen Linux-Distributionen abweichen.

Konfigurationsdateien für das System selbst und für Serverdienste
\begin{lstlisting}
/etc
\end{lstlisting}

Startskripte für Serverdienste
\begin{lstlisting}
/etc/init.d
\end{lstlisting}

Konfigurationsdateien für Netzwerkschnittstellen\label{ref:appendix_interfaces}
\begin{lstlisting}
/etc/network
\end{lstlisting}

Web-Root-Verzeichnis für den Apache2 Webserver
\begin{lstlisting}
/var/www
\end{lstlisting}

Konfigurationsdateien für den Nameserver BIND
\begin{lstlisting}
/etc/bind
\end{lstlisting}

Konfigurationsdateien für Postfix Mailserver
\begin{lstlisting}
/etc/postfix
\end{lstlisting}

Konfigurationsdateien für Apache2 Webserver
\begin{lstlisting}
/etc/apache2
\end{lstlisting}

Benutzer home-Verzeichnisse
\begin{lstlisting}
/home
\end{lstlisting}

Konfiguration zur lokalen DNS-Auflösung
\begin{lstlisting}
/etc/resolv.conf
\end{lstlisting}

Hostname des Servers
\begin{lstlisting}
/etc/hostname
\end{lstlisting}

Lokale Auflösung
\begin{lstlisting}
/etc/hosts
\end{lstlisting}

Lokale Benutzer und Grupppen
\begin{lstlisting}
/etc/passwd
/etc/shadow
/etc/group
\end{lstlisting}


%------------------------------------------------------------------------------
\subsection{Kommandos zur Paketverwaltung}
%------------------------------------------------------------------------------
Ubuntu-Server verwendet zur Verwaltung von Anwendungen das Debian-basierende
Paketverwaltungs-Tool Apt\footnote{Advanced Package Tool:
\url{http://wiki.debian.org/Apt}}. Es wird dazu verwendet Programme zu
installieren und zu verwalten. Es ist also nicht notwendig die Programme aus dem
Quellcode zu kompilieren. Sie sehr bequem auf ein Repository von Anwendungen
zugreifen und sehr einfach Programme installieren und entfernen. 

Anwendungen werden in Paketen (Packages) bereitgestellt. Die Installation
erfolgt aus Repositories von Archiv-Servern im Internet. Im Folgenden sind die wichtigsten
Kommandos zur Paketverwaltung\footnote{Paketverwaltung mit aptitude:
\url{http://wiki.ubuntuusers.de/aptitude}} kurz dargestellt:

Aktualisierung der in den Repositories verfügbaren Paketlisten
\begin{lstlisting}
aptitude update
\end{lstlisting}

Suchen von Anwendungen
\begin{lstlisting}
aptitude search <search1> [search2] ...
\end{lstlisting}

Installation eines gewünschten Paketes
\begin{lstlisting}
aptitude install <paket1> [paket2] ...
\end{lstlisting}

Löscht das Paket, die Konfigurationsdateien bleiben erhalten
\begin{lstlisting}
aptitude remove <paket>
\end{lstlisting}

Löscht das Paket inklusive der Konfigurationsdateien
\begin{lstlisting}
aptitude purge <paket>
\end{lstlisting}

Anzeigen von zusätzlichen Paketinformationen
\begin{lstlisting}
aptitude show <paket>
\end{lstlisting}
