%------------------------------------------------------------------------------
\section{Konfiguration des Webservers mit Apache2}
%------------------------------------------------------------------------------

Melden Sie sich am Server \textbf{srv2-0} mit dem Benutzernamen \textbf{admin2}
und dem Passwort \textbf{password2} an.

Installieren Sie nun den Apache-Webserver:
\begin{lstlisting}
admin2@srv2-0:~$ sudo aptitude install apache2
\end{lstlisting}

Danach erzeugen wir eine einfache HTML-Datei.
\begin{lstlisting}
admin2@srv2-0:~$ sudo vi /var/www/index.html
\end{lstlisting}
Natürlich sollten hier auch die Angaben für Ihren Arbeitsplatz gemacht werden. Hier wird wieder exemplarisch 
der Inhalt für srv2-0 wiedergegeben:
\begin{scriptsize}
\begin{lstlisting}
<html>
  <head>
    <title>beta.tklabor.site</title>
  </head>  
  <body>
    <h1>beta.tklabor.site</h1>
  </body>
</html>
\end{lstlisting}
\end{scriptsize}
Öffnen Sie einen Internet-Browser auf Ihrem Arbeitsplatz. Die Webseite sollte bereits über die IP-Adresse erreichbar sein. Geben Sie dazu
beispielsweise für srv2-0 als URL \textbf{http://192.168.1.3/} an.
