%------------------------------------------------------------------------------
\section{Webserver}
%------------------------------------------------------------------------------

%------------------------------------------------------------------------------
\subsection{Konfiguration des Webservers mit Apache2}
%------------------------------------------------------------------------------
Melden Sie sich am Server \textit{srv201} mit dem Benutzernamen \textit{admin2}
und dem Passwort \textit{password2} an.

Installieren Sie nun den Apache-Webserver:
\begin{lstlisting}
admin2@srv201:~$ sudo aptitude install apache2
\end{lstlisting}

Danach erzeugen wir eine einfache HTML-Datei.
\begin{lstlisting}
admin2@srv201:~$ sudo vi /var/www/index.html
\end{lstlisting}
Natürlich sollten hier auch die Angaben für Ihren Arbeitsplatz gemacht werden. Hier wird wieder exemplarisch 
der Inhalt für srv201 wiedergegeben:
\begin{scriptsize}
\begin{lstlisting}
<html>
  <head>
    <title>beta.tklabor.site</title>
  </head>  
  <body>
    <h1>beta.tklabor.site</h1>
  </body>
</html>
\end{lstlisting}
\end{scriptsize}
Öffnen Sie einen Internet-Browser auf Ihrem Arbeitsplatz. Die Webseite sollte bereits über die IP-Adresse erreichbar sein. Geben Sie dazu
beispielsweise für srv201 als URL\\
\url{http://10.174.26.201/} an.

%------------------------------------------------------------------------------
\subsection{Zugriffsschutz für Web-Inhalte}
%------------------------------------------------------------------------------
Der Zugriff auf Inhalte über HTTP ist standardmäßig für jeden erlaubt. Es kann
allerings notwendig sein, dass bestimmte Inhalte oder Dateien nur für einen
begrenzten Benutzerkreis zugänglich sein sollen. Im Folgenden Abschnitt wird auf
dem Webserver ein privates Verzeichnis angelegt. Der Zugang auf Dateien in
diesem Verzeichnis sind nur einem Benutzernamen und Passwort möglich.

Auf dem Webserver srv201 anmelden und für das Verzeichnis \texttt{/var/www} die 
Option \\
\texttt{AllowOverride None} auf \texttt{AuthConfig} ändern.

\begin{lstlisting}
admin2@srv201:~$ sudo vi /etc/apache2/sites-available/default
\end{lstlisting}

Die Datei sollte wie folgend geändert werden:

\begin{scriptsize}
\begin{lstlisting}
<VirtualHost *:80>
	ServerAdmin webmaster@localhost

	DocumentRoot /var/www
	<Directory />
		Options FollowSymLinks
		AllowOverride None
	</Directory>
	<Directory /var/www/>
		Options Indexes FollowSymLinks MultiViews
		AllowOverride AuthConfig
		Order allow,deny
		allow from all
	</Directory>
.
.
.
\end{lstlisting}
\end{scriptsize}
Die Änderungen werden mit einem Neustarten des \textit{Apache2} übernommen.

\begin{lstlisting}
admin2@srv201:~$ sudo service apache2 restart
\end{lstlisting}

Als nächstes wird eine Verzeichnisstruktur erstellt. Im Verzeichnis
\texttt{files} sind Dateien öffentlich zugänglich. Im Verzeichnis
\texttt{private} soll ein Zugang nur über Benutzername/Passwort möglich sein.

\begin{lstlisting}
admin2@srv201:~$ sudo mkdir -p /var/www/files/private/
\end{lstlisting}

Der Zugriff kann auf Verzeichnissebene über die Datei \texttt{.htaccess}
geregelt werden. Um das Verzeichnis \texttt{private} abzusichern wird die Datei
\texttt{.htaccess} wie folgt erstellt:

\begin{lstlisting}
admin2@srv201:~$ sudo vi /var/www/files/private/.htaccess
\end{lstlisting}

\begin{scriptsize}
\begin{lstlisting}
AuthType Basic
AuthName "This is a private area. Password Required"
AuthUserFile /etc/apache2/auth/password.file
require valid-user
\end{lstlisting}
\end{scriptsize}

In dieser Datei wird angegeben, dass Benutzern der Zugriff erlaubt wird, wenn
diese in einer Passwortdatei \texttt{password.file} angegeben sind. Die Datei
wird erzeugt und ein Benutzer \texttt{wwwuser1} mit dem Kennwort \texttt{user1}
angelegt.

\begin{lstlisting}
admin2@srv201:~$ sudo mkdir /etc/apache2/auth
admin2@srv201:~$ sudo htpasswd -c /etc/apache2/auth/password.file \
wwwuser1
\end{lstlisting}

Zum Test wird eine Textdatei im Verzeichnis \texttt{private} angelegt.

\begin{lstlisting}
admin2@srv201:~$ sudo echo "PRIVAT!" > \
/var/www/files/private/text.txt
\end{lstlisting}\

Testen Sie den Zugriff auf die Datei über
\url{http://10.174.26.201/files} und \url{http://10.174.26.201/files/private}

%------------------------------------------------------------------------------
\subsection{Aufgaben}
%------------------------------------------------------------------------------
\begin{enumerate}
  \item Beschreiben Sie den Aufbau eines \texttt{HTTP GET} auf die URL
  \url{http://10.174.26.201/index.html}
  \item Schauen Sie sich das Verhalten auf den geschützten Bereich im Logfile
  des Webservers in \texttt{/var/log/apache2/error.log}
  \item Fügen Sie einen \textit{CNAME Record} für den webserver so ein, dass die
  Webseite über \url{http://www.beta.tklabor.site} erreichbar wird.
\end{enumerate}