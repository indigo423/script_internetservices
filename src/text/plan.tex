\section{Planung und Netzwerkkonzept}
Der Versuchsaufbau wird im TK-Labor durchgeführt. Jede Lerngruppe
bestehend aus maximal drei Studierenden und erhält ein eigenes privates IPv4
Netz. Es können bis zu 16 Gruppen gebildet werden. Jede Gruppe kann 13
Netzwerkgeräte pro Subnetz einrichten. Die erste Adresse wird für den Router ins
Internet reserviert. Zusätzlich erhält jede Lerngruppe eine Sub-Domain für die
sie verantwortlich ist. Für diese Sub-Domains sind DNS-Server mit entsprechenden
Zonen einzurichten. Die Server sollen in den Zonen per DNS erreichbar sein. Die
IP-Adressen und Domains sind dabei wie folgt eingeteilt:

\begin{table}[!h]
  \centering
  \begin{tabular}{l l l l l l}
	\hline
	Netz-ID & Router & Srv1 & Srv2 & Broadcast & Sub-Domain \\
	\hline
	\textcolor{red}{192.168.1.0} & .1 & .2 & .3 & \textcolor{red}{.15} &
	beta.tklabor.site \\ \textcolor{red}{192.168.1.16} & .17 & .18 & .19 &
	\textcolor{red}{.31} & antares.tklabor.site \\
	\textcolor{red}{192.168.1.32} & .33 & .34 & .35 & \textcolor{red}{.47} &
	orion.tklabor.site \\
	\textcolor{red}{192.168.1.48} & .49 & .50 & .51 & \textcolor{red}{.63} &
	vogsphere.tklabor.site \\
	\hline
	\textcolor{red}{192.168.1.64} & .65 & .66 & .67 & \textcolor{red}{.79} &
	frogstar.tklabor.site \\
	\textcolor{red}{192.168.1.80} & .81 & .82 & .83 & \textcolor{red}{.95} &
	epun.tklabor.site \\
	\textcolor{red}{192.168.1.96} & .97 & .98 & .99 & \textcolor{red}{.111} &
	earth.tklabor.site \\
	\textcolor{red}{192.168.1.112} & .113 & .114 & .115 & \textcolor{red}{.127} &
	earth2.tklabor.site \\
	\hline
	\textcolor{red}{192.168.1.128} & .129 & .130 & .131 & \textcolor{red}{.143} &
	fallia.tklabor.site \\
	\textcolor{red}{192.168.1.144} & .145 & .146 & .147 & \textcolor{red}{.159} &
	magrathea.tklabor.site \\
	\textcolor{red}{192.168.1.160} & .161 & .162 & .163 & \textcolor{red}{.175} &
	lamuella.tklabor.site \\
	\textcolor{red}{192.168.1.176} & .177 & .178 & .179 & \textcolor{red}{.191} &
	kria.tklabor.site \\
	\hline
	\textcolor{red}{192.168.1.192} & .193 & .194 & .195 & \textcolor{red}{.207} &
	zarss.tklabor.site \\
	\textcolor{red}{192.168.1.208} & .209 & .210 & .211 & \textcolor{red}{.223} &
	rupert.tklabor.site \\
	\textcolor{red}{192.168.1.224} & .225 & .226 & .227 & \textcolor{red}{.239} &
	eadrax.tklabor.site \\
	\textcolor{red}{192.168.1.240} & .241 & .242 & .243 & \textcolor{red}{.255} &
	thesun.tklabor.site \\
	\hline
	\end{tabular}
	\caption{IP-Adressen für die Übungsgruppen}
	\label{tab:chap-labdocu-ipadressplan}
\end{table}

Ein zentraler Router stellt den Internetzugang für das Labornetzwerk zur
verfügung. Die Konfiguration des zentralen Routers soll mit den Studierenden
gemeinsam erarbeitet werden.

Beantworten sie die folgenden Fragestellungen anhand der Abbildung
\ref{fig:chap-labdocu-netzplan} auf Seite \pageref{fig:chap-labdocu-netzplan}.

\begin{itemize}
  \item Wie kann eine Kommunikation der Subnetze der Lerngruppen untereinander
  hergestellt werden?
  \item Wie kann ein zentraler Internetzugang realisiert werden? 
  \item Wie muss eine NAT-Konfiguration aussehen und was passiert hiert?
  \item Wie kann die Funktionsfähigkeit der Konfiguration auf Layer-3 überprüft
  werden?
  \item Wie muss die Konfiguration der Server als auch des zentralen Routers
  aussehen, damit DNS-Abfragen über den zentralen Router erfolgen?
  \item Diskutieren Sie Vor- und Nachteile eines zentralen DNS-Zugangs?
  \item Diskutieren Sie Vor- und Nachteile eines Internetzugangs mittels NAT? 
\end{itemize}

Der Versuchsaufbau findet im Raum C106 statt. Die entsprechende
Konfigurationen und die Verteilung der Domains und IP-Adressen im Labor ist
in der Abbildung \ref{fig:chap-labdocu-aufbau} auf Seite
\pageref{fig:chap-labdocu-aufbau} beschrieben.

