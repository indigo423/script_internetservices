%------------------------------------------------------------------------------
\section{Planung und Netzwerkkonzept}
%------------------------------------------------------------------------------
Der Versuchsaufbau wird im TK-Labor durchgeführt. Jede Lerngruppe
bestehend aus maximal drei Studierenden. Es können bis zu 14 Gruppen gebildet
werden. Jede Gruppe kann 2 Server einrichten. Zusätzlich erhält jede Lerngruppe
eine Sub-Domain für die sie verantwortlich ist. Für diese Sub-Domains sind DNS-Server mit entsprechenden
Zonen einzurichten. Die Server sollen später per DNS erreichbar sein. Die
IP-Adressen und Domains sind dabei wie folgt eingeteilt:

\begin{table}[!h]
  \centering
  \begin{tabular}{l l l l l l}
	\hline
	\textbf{Domain} & \textbf{Server 1} & \textbf{Server 2} \\
	\hline
	beta.tklabor.site & 10.174.26.200 (srv200) & 10.174.26.201 (srv201) \\
	antares.tklabor.site & 10.174.26.202 (srv202) & 10.174.26.203 (srv203) \\
	\hline
	orion.tklabor.site & 10.174.26.204 (srv204) & 10.174.26.205 (srv205) \\
	vogsphere.tklabor.site & 10.174.26.206 (srv206) & 10.174.26.207 (srv207) \\
	\hline
	frogstar.tklabor.site & 10.174.26.208 (srv208) & 10.174.26.209 (srv209) \\
	epun.tklabor.site & 10.174.26.210 (srv210) & 10.174.26.211 (srv211) \\
	\hline
	earth.tklabor.site & 10.174.26.212 (srv212) & 10.174.26.213 (srv213) \\
	earth2.tklabor.site & 10.174.26.214 (srv214) & 10.174.26.215 (srv215) \\
	\hline
	fallia.tklabor.site & 10.174.26.216 (srv216) & 10.174.26.217 (srv217) \\
	magrathea.tklabor.site & 10.174.26.218 (srv218) & 10.174.26.219 (srv219) \\
	\hline
	lamuella.tklabor.site & 10.174.26.220 (srv220) & 10.174.26.221 (srv221) \\
	kria.tklabor.site & 10.174.26.222 (srv222) & 10.174.26.223 (srv223) \\
	\hline
	zarss.tklabor.site & 10.174.26.224 (srv224) & 10.174.26.225 (srv225) \\
	rupert.tklabor.site & 10.174.26.226 (srv226) & 10.174.26.227 (srv227) \\
	\end{tabular}
	\caption{IP-Adressen für die Übungsgruppen}
	\label{tab:chap-labdocu-ipadressplan}
\end{table}

Ein zentraler Router stellt den Internetzugang für das Labornetzwerk zur
verfügung.

Beantworten sie die folgenden Fragestellungen anhand der Abbildung
\ref{fig:chap-labdocu-netzplan} auf Seite \pageref{fig:chap-labdocu-netzplan}.

%------------------------------------------------------------------------------
\subsection{Aufgabe}
%------------------------------------------------------------------------------
\begin{itemize}
  \item Wie kann ein zentraler Internetzugang realisiert werden?
  \item Wie muss eine NAT-Konfiguration aussehen und was passiert hiert?
  \item Wie kann die Funktionsfähigkeit der Konfiguration auf Layer-3 überprüft
  werden?
  \item Diskutieren Sie Vor- und Nachteile eines zentralen DNS-Zugangs?
  \item Diskutieren Sie Vor- und Nachteile eines Internetzugangs mittels NAT? 
\end{itemize}

Der Versuchsaufbau findet im Raum C106 statt. Die entsprechende
Konfigurationen und die Verteilung der Domains ist in der Abbildung
\ref{fig:chap-labdocu-aufbau} auf Seite \pageref{fig:chap-labdocu-aufbau}
beschrieben.

