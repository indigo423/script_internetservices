%------------------------------------------------------------------------------
\section{Konfiguration des Mailservers mit Postfix}
%------------------------------------------------------------------------------

Melden Sie sich am Server \textit{srv201} mit den bekannten Zugangsdaten an. 

Installieren Sie nun den Postfix-Mailserver:
\begin{lstlisting}
admin2@srv201:~$ sudo aptitude install postfix
\end{lstlisting}
Im Installationsverlauf wählen Sie \textbf{Internet Site} und als Mailname
\textit{mail}. Danach erfolgt die weitere Konfiguration des Servers. Editieren
Sie dazu die Datei \texttt{/etc/postfix/main.cf}.
\begin{scriptsize}
\begin{lstlisting}
admin2@srv201:~$ sudo vi /etc/postfix/main.cf
\end{lstlisting}
Folgende Einträge sollten beispielsweise für srv201 wie folgt angepasst werden:
\begin{lstlisting}
myhostname = mail.beta.tklabor.site
mydestination = beta.tklabor.site, 
                mail.beta.tklabor.site, 
                localhost.beta.tklabor.site, localhost
mynetworks = 127.0.0.0/8 [::ffff:127.0.0.0]/104 [::1]/128 10.174.26.0/24
\end{lstlisting}
\end{scriptsize}

Danach wird der Dienst neu gestartet.
\begin{lstlisting}
admin1@srv201 sudo service postfix restart
\end{lstlisting}
Damit ist die Konfiguration des Mailservers abgeschlossen, allerdings können keinerlei E-Mails zugestellt werden, da die so genannten MX-Records im Nameserver noch fehlen.

%------------------------------------------------------------------------------
\subsection{Nameserver-Einträge für den Mailserver}
%------------------------------------------------------------------------------
Melden Sie 
sich wieder am Server \textit{srv200} mit den bekannten Zugangsdaten an.
Editieren Sie die Zonen-Datei \texttt{/etc/bind/db.beta.tklabor.site}.
\begin{lstlisting}
admin1@srv200 sudo vi /etc/bind/db.beta.tklabor.site
\end{lstlisting}
Die Datei sollte nun beispielsweise für srv201 folgenden Inhalt aufweisen:
\begin{scriptsize}
\begin{lstlisting}
$TTL 3600
$ORIGIN beta.tklabor.site.
;
; forward lookup for beta.tklabor.site
;
@    IN   SOA  ns.beta.tklabor.site. root.beta.tklabor.site. ( 
				100		; Serial
				1m		; Refresh
				5m		; Retry
				30d		; Expire
				1h )	; Negative caching TTL
     IN   NS   ns.beta.tklabor.site.
ns   IN   A    10.174.26.200
@    IN   A    10.174.26.200
mail IN   A    10.174.26.201
\end{lstlisting}
\end{scriptsize}
Starten Sie den Nameserver neu:
\begin{lstlisting}
admin1@srv200 sudo service bind9 restart
\end{lstlisting}
Testen Sie die Namensauflösung mit folgendem Befehl:
\begin{scriptsize}
\begin{lstlisting}
admin1@srv200:~$ dig mx beta.tklabor.site

; <<>> DiG 9.7.1-P2 <<>> mx beta.tklabor.site
;; global options: +cmd
;; Got answer:
;; ->>HEADER<<- opcode: QUERY, status: NOERROR, id: 50085
;; flags: qr rd ra; QUERY: 1, ANSWER: 1, AUTHORITY: 1, ADDITIONAL: 2

;; QUESTION SECTION:
;beta.tklabor.site.	IN	MX

;; ANSWER SECTION:
beta.tklabor.site.	3600 IN	MX	10 mail.beta.tklabor.site.

;; AUTHORITY SECTION:
beta.tklabor.site.	1623 IN	NS	ns.beta.tklabor.site.

;; ADDITIONAL SECTION:
mail.beta.tklabor.site. 3600 IN A	10.174.26.201
ns.beta.tklabor.site. 1623	IN A	10.174.26.200

;; Query time: 2 msec
;; SERVER: 10.174.26.200#53(10.174.26.126)
;; WHEN: Thu Mar 10 10:16:42 2011
;; MSG SIZE  rcvd: 118
\end{lstlisting}
\end{scriptsize}

%------------------------------------------------------------------------------
\subsection{E-Mail-Versand}
%------------------------------------------------------------------------------
Melden Sie sich am Server \textit{srv201} mit dem Benutzernamen \textit{admin2}
und dem Passwort \textit{password2} an. Beispielsweise könnte \textit{admin2}
auf
\\
\textit{srv201.beta.tklabor.site} eine E-Mail an \\
\textit{admin2@beta.tklabor.site} mit folgenden Befehl schicken:

\begin{lstlisting}
admin2@srv201$ echo "Hallo Welt!" | mail admin2@beta.tklabor.site
\end{lstlisting}

Dieser könnte dann die E-Mail folgendermaßen lesen:
\begin{scriptsize}
\begin{lstlisting}
admin2@srv201:~$ mail
"/var/mail/admin2": 1 message 1 new
>N   1 admin2             Do Apr 10 10:32  15/789   
? 1
Return-Path: <admin2@beta.tklabor.site>
X-Original-To: admin2@beta.tklabor.site
Delivered-To: admin2@beta.tklabor.site
Received: from mail.beta.tklabor.site (unknown [10.174.26.201])
	by mail.beta.tklabor.site (Postfix) with ESMTP id E4A4221FE3
	for <admin2@beta.tklabor.site>; Thu, 10 Mar 2011 10:32:47 +0100 (CET)
Received: by mail.beta.tklabor.site (Postfix, from userid 1000)
	id 62D6721F41; Thu, 10 Mar 2011 10:32:44 +0100 (CET)
To: <admin2@beta.tklabor.site>
X-Mailer: mail (GNU Mailutils 2.1)
Message-Id: <20110310093244.62D6721F41@mail.beta.tklabor.site>
Date: Thu, 10 Mar 2011 10:32:44 +0100 (CET)
From: admin2@beta.tklabor.site (admin2)

Hallo Welt
? q
Saved 1 message in /home/admin2/mbox
Held 0 messages in /var/mail/admin2
admin2@srv201:~$ 
\end{lstlisting}
\end{scriptsize}

Damit ist diese Übung abgeschlossen.
