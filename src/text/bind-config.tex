%------------------------------------------------------------------------------
\section{DNS-Server mit BIND}
%------------------------------------------------------------------------------
Im Folgenden wird nun der Server \textbf{srv1} als Nameserver konfiguriert. Hier
wird wieder exemplarisch die Konfiguration von \textbf{srv1-0} wiedergegeben.
Die für Sie benötigten Werte entnehmen Sie bitte aus der Tabelle im Anhang.
Melden Sie sich wieder am Server \textbf{srv1} mit den bekannten Zugangsdaten
an. Danach wird nun die Software \textbf{bind} installiert.

\begin{lstlisting}
admin1@srv1-0$ sudo aptitude install bind9
\end{lstlisting}

Nach der Installation können wir nun die DNS-Zone konfigurieren:
\begin{lstlisting}
admin1@srv1-0$ sudo vi /etc/bind/named.conf.options
\end{lstlisting}
Die Datei sollte folgenden Inhalt aufweisen:
\begin{scriptsize}
\begin{lstlisting}
options {
    directory "/var/cache/bind";

    // If there is a firewall between you and nameservers you want
    // to talk to, you may need to fix the firewall to allow multiple
    // ports to talk.  See http://www.kb.cert.org/vuls/id/800113
  
    // If your ISP provided one or more IP addresses for stable 
    // nameservers, you probably want to use them as forwarders.  
    // Uncomment the following block, and insert the addresses replacing
    // the all-0's placeholder.

    forwarders {
      192.168.1.1;
    };

    auth-nxdomain no;    # conform to RFC1035
    listen-on-v6 { any; };
};
\end{lstlisting}
\end{scriptsize}

Damit werden Anfragen, die durch den lokalen Nameserver nicht auflösbar sind an den übergeordneten weitergegeben.

Nun erzeugen wir unsere Zonen-Datei, also beispielsweise
\textbf{/etc/bind/db.beta.tklabor.site}:
\begin{lstlisting}
admin1@srv-1$ sudo vi /etc/bind/db.beta.tklabor.site
\end{lstlisting}
Diese sollte dann folgenden Inhalt haben:
\begin{lstlisting}
$TTL 3600
$ORIGIN beta.tklabor.site.
;
; forward lookup for beta.tklabor.site
;
@    IN   SOA  ns.beta.tklabor.site. root.beta.tklabor.site. ( 
                 100000000 15m 5m 30d 1h 
               )
     IN   NS   ns.beta.tklabor.site.

ns   IN   A    192.168.1.1
\end{lstlisting}
\normalsize Diese Zonen-Datei muss nun noch in der Konfiguration inkludiert werden. Also editieren wir nun die Datei \textbf{/etc/bind/named.conf.local}:
\begin{lstlisting}
admin1@srv1$ sudo vi /etc/bind/named.conf.local
\end{lstlisting}
Diese sollte dann folgendermaßen aussehen:
\begin{lstlisting}
//
// Do any local configuration here
//

// Consider adding the 1918 zones here, if they are not used in your
// organization
//include "/etc/bind/zones.rfc1918";

zone "tknet02.informatik.hs-fulda.de" IN {
  type master;
  file "/etc/bind/db.tknet02";
  allow-transfer { none; };
  notify no;
};
\end{lstlisting}
Jetzt stoppen und starten wir den entsprechenden Dienst:
\begin{lstlisting}
admin1@srv1$ sudo service bind9 restart
\end{lstlisting}

In der Regel sollte es nun möglich sein den Nameservers der Domäne \\ \textbf{tknet02.informatik.hs-fulda.de} zu ermitteln.

\begin{lstlisting}
admin1@srv1:~$ dig ns tknet02.informatik.hs-fulda.de

; <<>> DiG 9.7.1-P2 <<>> ns tknet02.informatik.hs-fulda.de
;; global options: +cmd
;; Got answer:
;; ->>HEADER<<- opcode: QUERY, status: NOERROR, id: 4257
;; flags: qr rd ra; QUERY: 1, ANSWER: 1, AUTHORITY: 0, ADDITIONAL: 1

;; QUESTION SECTION:
;tknet02.informatik.hs-fulda.de.	IN	NS

;; ANSWER SECTION:
tknet02.informatik.hs-fulda.de.	3600 IN	NS	ns.tknet02.informatik.hs-fulda.de.

;; ADDITIONAL SECTION:
ns.tknet02.informatik.hs-fulda.de. 3600	IN A	10.174.26.202

;; Query time: 16 msec
;; SERVER: 10.174.26.126#53(10.174.26.126)
;; WHEN: Thu Mar 10 09:43:44 2011
;; MSG SIZE  rcvd: 81
\end{lstlisting}
